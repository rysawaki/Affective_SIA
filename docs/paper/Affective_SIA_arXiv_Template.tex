\documentclass[11pt, a4paper]{article}

% --- UNIVERSAL PREAMBLE BLOCK ---
\usepackage[a4paper, top=2.5cm, bottom=2.5cm, left=2cm, right=2cm]{geometry}
\usepackage{fontspec}
\usepackage[english, bidi=basic, provide=*]{babel}
\babelprovide[import, onchar=ids fonts]{english}
\babelfont{rm}{Noto Sans}

% --- Packages ---
\usepackage{amsmath, amssymb, amsfonts}
\usepackage{graphicx}
\usepackage{hyperref}
\usepackage{bm}
\usepackage{caption}
\usepackage{subcaption}
\usepackage{physics}
\usepackage{booktabs}
\usepackage{float}

% --- Page Setup ---
\setlength{\parskip}{0.5em}
\setlength{\parindent}{0pt}

% --- Meta Data ---
\title{
    \textbf{Affective Active Inference: \\
    From Trace to Shared Engram and Narrative Identity Formation}
}

\author{
    Ryota Sawaki \\
    \textit{Department of Computational Psychiatry} \\
    \texttt{rysawaki@gmail.com}
}

\date{\today}

\usepackage{hyperref}

\hypersetup{
    colorlinks=true,       
    linkcolor=blue,        
    citecolor=blue,        
    urlcolor=blue,        
    pdftitle={SIA Model},  
    pdfauthor={Ryota Sawaki} 
}

% ---  ---

\begin{document}
\maketitle

\begin{abstract}
Human identity does not emerge from isolated perception–action cycles but from accumulated affective experiences that shape long-term self–other attribution. We propose the Self-Imprint Attribution (SIA) model, an extension of Active Inference that transforms prediction errors into imprint traces, which progressively evolve into vectorized affective states. These affect vectors modulate future policy selection, enabling the emergence of narrative identity beyond instantaneous error minimization.

We conducted multi-agent simulations in a dyadic affective meaning exchange task within a continuous embedding space, where agents share signals representing interpretation, affect, and action. We show that Shared Engram—a distributed memory structure—emerges only when three conditions are simultaneously satisfied:
(1) synchronized affective depth,
(2) aligned action direction, and
(3) mutual self-attribution confidence.

Crucially, sensitivity analysis reveals that agents with high trace sensitivity (α=2.0) develop identities approximately 5× stronger than agents with α=0.1, demonstrating that computational vulnerability is not a flaw but a prerequisite for identity formation and social resilience.

These results provide the first computational account of how affective resonance gives rise to narrative identity within an Active Inference framework, suggesting new directions for affective AI, trauma modeling, and interactive AGI architectures.
\end{abstract}

\textbf{Code Availability:} The simulation code and data specifically used to generate the results presented in this study are available at \url{https://github.com/rysawaki/Affective_SIA}.
% =========================
\section{Introduction}
% =========================
Why do we hold onto painful memories? In standard Reinforcement Learning (RL) and Active Inference, prediction errors (surprisal) are costs to be minimized or ignored once learning is complete \cite{parr2019generalised}. However, in human psychology, unresolved errors often form the core of one's personality—a phenomenon known as trauma or core belief \cite{vanderkolk2014body}.

We propose that these errors are not waste products but the raw material of identity. Drawing on theories of narrative identity \cite{mcadams2001psychology, ricoeur1991narrative} and interoceptive inference \cite{seth2013interoceptive}, we introduce \textbf{Affective Active Inference}, where:
\begin{enumerate}
    \item \textbf{Trace:} Discrepancies are imprinted physically \cite{vanderkolk2014body}.
    \item \textbf{Affect:} Traces are interpreted into qualitative vectors (e.g., Sorrow, Hope) \cite{damasio1999feeling}.
    \item \textbf{Identity:} These vectors are integrated over time through shared resonance.
\end{enumerate}

% =========================
\section{Computational Model}
% =========================
The SIA agent operates on a cycle of five phases. The core mechanism is the \textit{Attribution Gate}, which determines ownership of experience.

\subsection{Self-Attribution and Trace}
The probability that an experience $E$ belongs to the self, $P(Self|E)$, is defined by:
\begin{equation}
    P(Self|E) = \sigma \left( -\|E - S\| + \alpha \|T\| + \beta \|Act_{prev}\| \right)
\end{equation}
where $T$ is the accumulated trace, and $\alpha$ is the \textbf{Trace Sensitivity} parameter. A higher $\alpha$ implies that the agent is more likely to attribute painful discrepancies to itself (internalization).

\subsection{Identity Integration via Shared Resonance}
Identity $I(t)$ is not a static variable but a historical integral. It grows only when a \textit{Shared Engram} is formed with another agent:
\begin{equation}
    I(t+1) = I(t) + \eta \cdot \text{Shared}(t) \cdot \mathbf{A}(t)
\end{equation}
where $\mathbf{A}(t)$ is the affective vector. The shared resonance is strictly defined as:
\begin{equation}
    \text{Shared}(t) = P_1 P_2 \cdot \cos(\theta_{act}) \cdot \cos(\theta_{aff}) \cdot \min(\|\mathbf{A}_1\|, \|\mathbf{A}_2\|)
\end{equation}
This ensures that identity is formed only when both agents share both the \textit{direction of action} and the \textit{quality of affect}.

% =========================
\section{Simulation Results}
% =========================
We conducted two experiments to validate the theory.

\subsection{Dynamics of Narrative Identity Formation}
Figure \ref{fig:identity} illustrates the time evolution of an agent interacting with a partner after a traumatic event ($t=50$).
\begin{itemize}
    \item \textbf{Phase 1 (Trauma):} The agent experiences a shock, generating a large Trace.
    \item \textbf{Phase 2 (Genesis):} The trace is converted into Affect (Meaning).
    \item \textbf{Phase 3 (Resonance):} Upon encountering a partner ($t=100$), the Shared Engram (Green area) emerges.
    \item \textbf{Result:} The Identity ($I$, Black line) begins to accumulate only after resonance occurs, supporting the hypothesis: ``I become who we were.''
\end{itemize}

\begin{figure}[H]
    \centering
    % Ensure identity_formation.png is present in the same directory
    \includegraphics[width=0.85\textwidth]{identity_formation.png}
    \caption{Simulation of Narrative Identity Formation. The Identity (bottom panel) grows only when Shared Resonance (middle panel, green) is active.}
    \label{fig:identity}
\end{figure}

\subsection{Sensitivity Analysis: The Paradox of Vulnerability}
To investigate the role of individual differences, we performed a parameter sweep on Trace Sensitivity $\alpha$ ($0.1 \leq \alpha \leq 2.0$).
As shown in Figure \ref{fig:sensitivity}, there is a strong positive correlation between $\alpha$ and the final magnitude of Identity.

\begin{figure}[H]
    \centering
    % Ensure sensitivity_analysis.png is present in the same directory
    \includegraphics[width=0.7\textwidth]{sensitivity_analysis.png}
    \caption{Sensitivity Analysis. Agents with higher sensitivity to trace ($\alpha$) form stronger identities. Vulnerability fosters resilience.}
    \label{fig:sensitivity}
\end{figure}

% =========================
\section{Discussion}
% =========================
The results present a counter-intuitive insight: \textbf{Vulnerability is Strength.}
Agents that easily ignore traces ($\alpha \approx 0$) minimize immediate costs but fail to accumulate the affective resources necessary for deep social resonance. Conversely, agents that retain traces ($\alpha > 1.0$) suffer more initially but use that suffering as fuel for creative action and identity formation.

This suggests that in the context of General Intelligence, "forgetting" is not always optimal. The ability to be scarred (Imprint) is the engine of becoming someone (Identity).

% =========================
\section{Conclusion}
% =========================
We demonstrated that Narrative Identity can be computationally modeled as an integral of shared affective history. The SIA model provides a new mathematical bridge between clinical concepts of trauma and engineering concepts of AGI \cite{parr2019generalised, vanderkolk2014body}.

% --- Bibliography ---
\bibliographystyle{unsrt}
\bibliography{references}


\end{document}\begin{figure}
    \centering
    \includegraphics[width=0.5\linewidth]{identity_formation.png}
    \caption{Enter Caption}
\begin{figure}
        \centering
        \includegraphics[width=0.5\linewidth]{sensitivity_analysis.png}
        \caption{Enter Caption}
        \label{fig:placeholder}
    \end{figure}
        \label{fig:placeholder}
\end{figure}
