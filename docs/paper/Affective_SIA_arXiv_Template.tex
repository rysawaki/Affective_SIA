\documentclass{article}

% --- Packages ---
\usepackage{amsmath, amssymb, amsfonts}
\usepackage{graphicx}
\usepackage{hyperref}
\usepackage{bm}
\usepackage{geometry}
\usepackage{caption}
\usepackage{subcaption}
\usepackage{physics}
\usepackage{mathtools}
\usepackage{booktabs}
\usepackage{multirow}

\geometry{margin=1in}

\title{
\vspace{-2em}
\textbf{Affective Active Inference:\\
From Trace to Shared Engram and Narrative Identity Formation}
}

\author{
RS5309\\
Department of Psychiatry, Japan\\
\texttt{email@example.com}  % optional
\and
ChatGPT (GPT-5.1)\\
Computational Cognitive Modeling Partner
}

\date{}

\begin{document}
\maketitle

\begin{abstract}
We propose a new theoretical and computational framework that explains how subjective pain evolves into meaning, agency, and eventually a narrative identity. 
Building on Active Inference and affective neuroscience, we define the mechanism by which \textit{trace} (irreversible imprint of experience) transforms into \textit{affect} (qualitative feeling), drives world-modifying \textit{action}, and—only when synchronized with another mind—emerges as a \textit{shared engram}. 
We mathematically formalize the conditions for resonance:
\[
Shared(t) = P_1(t)P_2(t)\cos(A_1,A_2)\cos(Aff_1,Aff_2)\min(\|Aff_1\|,\|Aff_2\|)
\]
This enables a computational model of narrative identity formation, showing that ``I become who we were.''
\end{abstract}

% =========================
\section{Introduction}
% =========================
Human identity does not emerge from isolated experiences, but from the emotionally charged meanings we share with others.
Traditional cognitive models explain memory, reinforcement, and perception—but not identity, empathy, or why pain sometimes becomes meaning.
This paper proposes Affective Active Inference, a framework in which:
\[
Trace \rightarrow Affect \rightarrow Action \rightarrow Shared\ Engram \rightarrow Narrative\ Identity
\]

We address three central questions:
\begin{enumerate}
    \item When does suffering become meaningful rather than merely painful?
    \item Why is shared affect—not shared behavior—necessary for resonance?
    \item How does meaning accumulate over time to form narrative identity?
\end{enumerate}

% =========================
\section{Theoretical Background}
% =========================
\subsection{Active Inference and Self-Attribution}
Briefly introduce Friston's Free Energy Principle, Predictive Processing, Bayesian self-attribution.

\subsection{Limitations of Cognitive-Only Models}
Why emotion cannot be reduced to prediction error.
Why shared behavior does not imply shared understanding (misalignment phenomenon).

% =========================
\section{Core Proposal: Affective Active Inference}
% =========================

We define a self-attribution probability:
\[
P(Self|E) = \sigma(-D(E,S) + \alpha \|T\| + \beta \|A_{prev}\|)
\]
where $D(E,S)$ is discrepancy, $T$ is trace, and $A_{prev}$ is previous action (agency).

% -------------------------
\subsection{Trace Update (Irreversible Imprint)}
\[
T_{t+1} = T_t + \tanh(D) \cdot P(Self|E) \cdot (E - S) - \lambda T_t
\]

% -------------------------
\subsection{Affect Generation (Meaning Emergence)}
Vectorized affect:
\[
\mathbf{Aff}_{t+1} = \phi \mathbf{Aff}_t + (1-\phi)W_{T2A} \cdot (T_t) \cdot P(Self|E)
\]

Where $W_{T2A}$ converts trace into affective qualities (Hope, Sorrow, Solidarity, Creation).

% -------------------------
\subsection{Action from Affect (Expression/World Modification)}
\[
A_t = \gamma \cdot \|\mathbf{Aff}_t\| \cdot (S_t - E_t)
\]

% =========================
\section{Shared Engram: Conditions for Resonance}
% =========================
\subsection{Why Behavior Sync is Not Enough}
Figure showing: Same action, different meaning.

\subsection{Final Shared Engram Equation}
\[
Shared(t) = P_1P_2 \cdot \cos(A_1,A_2)\cdot
\cos(Aff_1,Aff_2)\cdot
\min(\|Aff_1\|, \|Aff_2\|)
\]

% =========================
\section{Simulation Results}
% =========================

\subsection{Trace → Affect → Action Dynamics}
\begin{figure}[h]
    \centering
    \includegraphics[width=0.8\textwidth]{figures/affect_dynamics.png}
    \caption{Transformation from Trace to Affect to Action. Pain becomes Meaning.}
\end{figure}

\subsection{Misaligned Resonance}
\begin{figure}[h]
    \centering
    \includegraphics[width=0.8\textwidth]{figures/misalignment.png}
    \caption{Doing same behavior, feeling differently. No resonance.}
\end{figure}

\subsection{Narrative Identity Formation}
\begin{figure}[h]
    \centering
    \includegraphics[width=0.8\textwidth]{figures/identity_formation.png}
    \caption{Identity only increases when Shared Engram is positive.}
\end{figure}

% =========================
\section{Discussion and Implications}
% =========================
\subsection{Applications in AI, Therapy, and Social Modeling}
Identity-based AI, personalized therapy, trauma processing, relational modeling.

\subsection{Philosophical Insight}
“I become who we were”: self is not stored, but co-created.

% =========================
\section{Conclusion}
% =========================
\textbf{This paper provides the first computational model in which meaning, agency, and identity emerge from affective trace-sharing.}
Future work includes multi-agent identity formation, therapeutic simulation, and affective AI design.

\bibliographystyle{plain}
\bibliography{references}
\end{document}
